\documentclass[12pt]{beamer}
%% Remove draft for real article, put twocolumn for two columns
\usetheme{CambridgeUS}
\usecolortheme{wolverine}
\usepackage[utf8]{vietnam}
\usepackage[style=alphabetic, backend=biber]{biblatex}
\addbibresource{bibliography.bib}
\setbeamertemplate{theorems}[numbered] 
\usepackage{physics}
\newtheorem{axiom}{Axiom}

%% commentary bubble
\newcommand{\SV}[2][]{\sidenote[colback=green!10]{\textbf{SV\xspace #1:} #2}}

%% Title 
\title{ Quantum Computation \\ Introduction }
\author{Truong-Son Van}
%\affil[1]{Institute}
\institute{Fulbright University Vietnam Reading Group}
\date{September 26, 2022}

\begin{document}

\frame{\titlepage}

\begin{frame}
    \frametitle{Logistics}
    Info (dates, speakers, references) is hosted on: \url{https://www.tsvan.xyz/reading.html}
    
    Follow John Preskill's notes 

    \url{http://theory.caltech.edu/~preskill/ph219/ph219_2021-22.html}

    Helpful videos:
    \begin{enumerate}
        \item Preskill's class: \url{https://www.youtube.com/playlist?list=PL0ojjrEqIyPy-1RRD8cTD_lF1hflo89Iu}
        \item UC Berkeley Vazirani's class: \url{https://www.youtube.com/playlist?list=PLXEJgM3ycgQW5ysL69uaEdPoof4it6seB}
    \end{enumerate}
\end{frame}

\begin{frame}
    \frametitle{ Basics: Complex Vector Space}
    A \emph{complex vector space} is a non-empty set $\mathbb{V}$, whose elements we call vectors, with three 
    operations
    \begin{enumerate}
        \item Addition $+: \mathbb{V} \times \mathbb{V} \to \mathbb{V}$
        \item Negation $-: \mathbb{V}\to \mathbb{V}$
        \item Scalar multiplication: $\cdot : \mathbb{C} \times \mathbb{V} \to \mathbb{V}$
    \end{enumerate}
    and a distinguished element called \emph{zero vector $0\in \mathbb{V}$}.
    The operations above obey usual rules with the scalar multiplication obeys rules for complex numbers.

    In QM, vectors will be donoted by $\ket{\varphi}$.

    \begin{example}
        $\mathbb{C}^n$, space of complex polynomials $\mathbb{C}[x]$, 
        space of complex-valued square integrable functions $L^2(\mathbb{R};\mathbb{C})$.
    \end{example}

    (Un)fortunately, we do have to work with complex numbers...
\end{frame}

\begin{frame}
    \frametitle{ Basics: Inner Product}
    \begin{definition}
        Given a vector space $\mathbb{V}$, an \emph{inner product } (dot product or scalar product)
        is a function
        \begin{equation*}
            \langle \cdot, \cdot \rangle : \mathbb{V}\times \mathbb{V} \to \mathbb{C}
        \end{equation*}
        that satisfies
        \begin{enumerate}
            \item $\langle V, V\rangle \geq 0$ and $\langle V, V \rangle = 0$ iff $V = 0$.
            \item $\langle V_1 + V_2, V_3 + V_4 \rangle = \langle V_1 , V_3 \rangle +
                                                \langle V_2 , V_4 \rangle \,.$
            \item $\langle c V_1, V_2\rangle = c \langle V_1, V_2 \rangle $
                and
                $ \langle V_1, c V_2 \rangle = \overline{c} \langle V_1, V_2 \rangle$.
            \item $\langle V_1, V_2 \rangle = \overline{ \langle V_2, V_1 \rangle  }$.
        \end{enumerate}
    \end{definition}
\end{frame}


\begin{frame}
    \frametitle{ Basics: Hilbert Space}
    \begin{definition}
        A Hilbert space is a complete vector space with an inner product.
    \end{definition}
    \begin{example}
        \begin{enumerate}
            \item $\mathbb{C}^n$ wiht the usual inner product
            \item $L^2(\mathbb{R};\mathbb{C})$ where
                \begin{equation*}
                    \langle f, g \rangle = \int \overline{f} g 
                \end{equation*}
        \end{enumerate}
        Fact: finite dimensional Hilbert spaces are isomorphic to $\mathbb{R}^n$.
        This concept is more useful in infinite dimensional spaces.
    \end{example}

\end{frame}

\begin{frame}
    \frametitle{ Basics: Hilbert Space}
    \begin{definition}
        Given a Hilbert space $\mathcal{H}$,
        a hermitian operator is a mapping $A:\mathcal{H} \to \mathcal{H}$ such that
        $A = A^\dag$, where $A^\dag: \mathrm{Dom(A^\dag)}\subset \mathcal{H} \to \mathcal{H}$ is defined by the following relation
        \begin{equation*}
            \langle A x , y \rangle = \langle x , A^\dag y \rangle \,.
        \end{equation*}
    \end{definition}
    \begin{itemize}
        \item For a bounded operator or in finite dimensions, hermitian is the same with self-adjoint.
    Issues come up when we deal with unbounded operators in  infinite dimensions.
        \item We mostly deal with finite dimensions, however. So this is not an issue. Just get used
    to the terminology of physicists.
    \end{itemize}

\end{frame}

\begin{frame}
\frametitle{Basics: Quantum Mechanics \pause (đểu)}
We approach this from the axiomatic point of view. 

\begin{axiom}[States]
    A state is a complete description of a physical system. 
    In quantum mechanics, a state is a ray in a Hilbert space.
\end{axiom}

\end{frame}



\printbibliography 
%\bibliography{refs}
%\bibliographystyle{halpha-abbrv}


\end{document}
